\documentclass[11pt]{article}
%Gummi|065|=)
\title{\textbf{TP4 - Arbres}}
\author{Louis Caron\\
		Océane Dubois}
\date{}
\begin{document}

\maketitle
\newpage

\section{Partie A : Représentation 0 - ABR}

Dans cette partie le but est de représenter un dictionnaire sous forme d'ABR, ou chaque noeud comporte un mot. Le fils gauche de ce mot à pour clé un mot plus petit selon l'ordre alphabétique et le fils droit est composé d'un mot qui est plus grand que la clé du noeud père, selon l'ordre alphabétique. 

Nous avons implémenté 2 structures pour coder ces foctions : Arbre, qui pointe toujours sur la racine de l'arbre et DicoABR qui est une cellule de l'arbre. 

Ainsi la fonction initDico, permet de d'allouer de la mémoire à la structure Arbre et à la structure DicoABR et de l'initialiser tel que Arbre pointe sur Dico ABR.

Cette fonction est en O(1) car ne réalise que des assignations

La fonction ajoutMot permet d'insérer un nouveau mot dans l'ABR que nous avons créé précédement. Cette fonction devra au maximum parcourir toute la hauteur de l'arbre, on sera donc en O(h).


La fonction rechercherMot permet de rechercher dans tout l'arbre si un mot est présent, elle devra également au maximum parcourir toute la hauteur de l'arbre. Sa compléxité est donc en O(h).

Nous avons également implémenté la fonction m


\end{document}

