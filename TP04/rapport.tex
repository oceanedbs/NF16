\documentclass[11pt]{article}
%Gummi|065|=)
\title{\textbf{TP4 - Arbres}}
\author{Louis Caron\\
		Océane Dubois}
\date{}
\begin{document}

\maketitle
\newpage

\section{Partie A : Représentation 0 - ABR}

Dans cette partie le but est de représenter un dictionnaire sous forme d'ABR, ou chaque noeud comporte un mot. Le fils gauche de ce mot à pour clé un mot plus petit selon l'ordre alphabétique et le fils droit est composé d'un mot qui est plus grand que la clé du noeud père, selon l'ordre alphabétique. 

Nous avons implémenté 2 structures pour coder ces foctions : Arbre, qui pointe toujours sur la racine de l'arbre et DicoABR qui est une cellule de l'arbre. 




\end{document}

