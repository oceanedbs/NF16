\documentclass[11pt]{report}
\usepackage{pgf}
\usepackage{tikz}
\usetikzlibrary{arrows,automata}
\usepackage[utf8]{inputenc}
 \usepackage{listings}
 \usepackage{color}
 \usepackage{fancyhdr}
 \usepackage{graphicx}
 \usepackage{amsmath}
\pagestyle{fancy}
 \usepackage{comment} 

 \lhead{Louis Caron \\ Océane DUBOIS}
 \rhead{NF16 - TP03 - Listes chainées}
 \rfoot{}



\definecolor{dkgreen}{rgb}{0,0.6,0}
\definecolor{gray}{rgb}{0.5,0.5,0.5}
\definecolor{mauve}{rgb}{0.58,0,0.82}

\lstset{frame=tb,
  language=c,
  aboveskip=3mm,
  belowskip=3mm,
  showstringspaces=false,
  columns=flexible,
  basicstyle={\small\ttfamily},
  numbers=none,
  numberstyle=\tiny\color{gray},
  keywordstyle=\color{blue},
  commentstyle=\color{dkgreen},
  stringstyle=\color{mauve},
  breaklines=true,
  breakatwhitespace=true,
  tabsize=3
}

 
%Gummi|065|=)
\title{\textbf{NF16 - TP03 - Listes chainées}
\author{Louis Caron \\ Océane DUBOIS\\}
\date{}}

\begin{document}

\maketitle

\newpage

\section{Presentation}

Le but de se TP est de se familiariser avec les listes chaînées. On cherchera donc à implémenter différentes fonctions permettant à l'utilisateur de créer des élements, créer des listes chainées, insérer des élements dans ses listes, supprimer des élements de ces listes, supprimer des listes, supprimer des elements, fusioner des listes ensemble .... 

En C il n'existe pas de type défini permettant de gérer des listes, contrairement à d'autres laguages de plus hauts niveaux. 

C'est donc notre but de créer d'abord des élements puis des listes composées de ces différents élements.

\section{Fonctions implementées}

Nous avons commencé par implementer la structure Element comme suivant : 

\begin{lstlisting}
struct Element {
	char valeur[20];
	struct Element* suivant;
	struct Element* precedent;
};
typedef struct Element T_Element;
\end{lstlisting}
On créée donc une structure composée d'une valeur de type chaine de caractères (de 20 caractères max) et de 2 pointeurs vers les Elements précédents et suivant dans la liste.

Ensuite nous avons créé la structure Liste :

\begin{lstlisting}
struct Liste{
	int taille;
	struct Element* tete;
	struct Element* queue;
};
typedef struct Liste T_Liste;

\end{lstlisting}

Cette structure est composée d'un entier pour connaître sa taille et de 2 pointeurs vers les élements tête et queue de la liste. 

Nous avons ensuite implémenté les 10 fonctions suivantes : 
\begin{lstlisting}
T_Element *creerElement(char *val);

T_Liste *creerListe();

int insererElement(T_Liste *list, char *val);

T_Element *rechercherElement(T_Liste *list, char *val);

int supprimerElement(T_Liste *list, char *val);

int supprimerListe(T_Liste *list);

T_Liste *fusionnerListes(T_Liste *list1, T_Liste *list2);

int afficher(T_Liste *list);

int menu();

int tableau_vide(T_Liste* tab[]); 
 
\end{lstlisting}

La fonction creerElement permet d'initialiser une structure de type T\_Element en initialisant valeur avec la chaine de caractère passée en paramètres, les pointeurs suivant et précédents sont initialisés à NULL. Cette fonction retourne un pointeur sur l'élement créé.

\medskip

La fonction creerListe permet d'inistialiser une structure de type T\_Liste. Sa taille est initialisée à 0 et les pointeurs tête et queue sont initialisés à NULL comme la liste est vide. La fonction retourne un pointeur vers la liste créée. 

\medskip

La fonction insererElement permet de créer l'element avec la valeur passée en paramètre et de l'insérer dans la liste en respectant l'ordre alphabetique. Dans cette fonction, on commence par créer un élement avec la fonction creerElement, puis on vérifie par rapport à l'ordre alphabetique si l'insertion se fait en début de liste, en fin de liste ou au milieu de la liste. Puis on réalise l'insertion en modifiant les pointeur précédent et suivant des élements avant et après la position où l'on souhaite insérer l'élément. 
Si l'insertion est réalisée correctement on retourne 0 et si l'insertion n'est pas réalisée correctement on retourne -1.

\medskip

La fonction rechercherElement permet de rechercher dans une liste donnée un élement avec une valeur donnée. Dans cette fonction, on vérifie tout d'abord si l'élement est plus petit que la tête ou plus grand que la queue de la liste, l'élément n'est donc pas dans la liste, sinon on parcours la liste pour chercher l'élément, dès qu'on arrive à une valeur plus grande que la valeur recherchée on sort de la boucle. La fonction retourne le pointeur vers l'élément recherché ou -1 sinon. 

\medskip

La fonction supprimerElement permet de supprimer un élément en fonction de sa valeur dans une liste donnée. Cette fonction appelle d'abord la fonction rechercherElement et si la fonction rechercherElement retourne un pointeur non nul, on vérifie si on est dans le cas où l'élément retourné est la tête de liste, la queue de la liste ou au milieu de la liste. 
Si la suppression est faire correctement on retourne 0 sinon -1. 

\medskip

La fonction supprimerListe permet de supprimer la liste passée en paramètre. Si la liste passée n'est pas nulle. On parcours toute la liste pour libérer les élements puis on libère la mémoire prise par la liste. 

\medskip

La fonction fusionerListe permet de fusionner 2 listes pour n'avoir qu'une seule liste à la fin, les 2 listes elles sont supprimées. On commence par vérifier si une des 2 listes est vide, si c'est le cas, on retourne la liste non vide, sinon on parcours les éléments des listes 1 par 1, on vérifie si un élément n'es pas en double dans l'autre liste puis on l'ajoute à la liste résultat et on supprimer cet élement de la liste source. Cette fonction retourne un pointeur vers le résultat de la fusion des 2 listes. 

\medskip

La fonction afficherListe permet d'afficher tous les élements d'une liste passée en paramètre. Cette fonction parcours simplement tous les éléments de la liste et les affiche avec leur position dans la liste. 

\medskip

La fonction menu permet d'afficher le menu des 8 différentes action que l'utilisateur peur réaliser et retourne le numéro de la fonction choisie par l'utilisateur. 

\medskip

La fonction tableau\_vide vérifie si le tableau passé en paramètre, cela nous sert pour voir dans le programme principal si des listes ont été créées. 

\medskip

Lorsque l'on quitte le programme, nous qvons voulu nous assurer que tous les malloc soient bien free, pour cela nous n'avons pas creer de fonctions mais avons mis dans le main une boucle s'assurant que toutes les listes soient bien free avant de quitter (on appelle supprimerListe). On a alors une complexite de O(\sum\limits_{i=1}^TAILLE tailleListe ) pour cette boucle.

\medskip

Afin de pouvoir gérer nos différentes listes, nous avons décidés de stocker nos différents pointeurs dans un tableau de taille 20. On demandera ainsi à l'utilisateur sur quelle liste il souhaite réaliser tel ou tel action à chaque fois. Les listes seront désignées par leur indice de position dans le tableau. Lorsque la limite du nombre de liste est atteinte et que l'utilisateur veut en créer plus, on lui demandera alors de supprimer une ancienne liste pour libérer une case du tableau. nous avons choisi cette structure de données car elle est simple à mettre en oeuvre et très efficace. Nous aurions pu créer une liste doublement chainée contenant toutes les listes (leurs pointeurs) mais cela aurait demandé la mise en place de fonctions adaptée pour supprimer, ajouter...etc des listes alors que le tableau présentait l'avantage d'être directement utilisable sans nécéssiter beaucoup de code supplémentaire. A noter que nous aons limité la taille du tableau à 20 ici, mais les fonctions restent valide avec un tableau beaucoup plus grand!

\section{Complexité}

Nous allons maintenant détailler la complexité de chaque fonction.

\medskip

La fonction creerElement n'est composée d'aucune boucle. Elle est donc en O(1).
\medskip
La fonction creerListe, tout comme la fonction creerElement ne comporte pas de boucles ou d'appels récursifs donc elle est en O(1).
\medskip
La fonction insererElement :
\begin{itemize}
	\item si la liste est vide, l'insertion se fait en O(1) car il n'y a que des opérations d'affectations.
	\item si la liste est non vide on parcours la liste, si la valeur à inséré est égale à une de la liste on ne fait rien, sinon, si l'insertion se fait en tête de liste on est en O(1). Si l'élément doit être insérer en fin de liste, alors on est en O(1) car nous avons aussi prit en compte ce cas particulier (afin de réduire la complexité de nos autres fonctions faisant appel à celle-ci). Dans tous les autres cas, on parcourt la liste afin de trouver l'emplacement où insérer l'élément, on a donc O(n-1) dans le pire de ces cas là, c'est à dire O(n).
	
\end{itemize}


\medskip

La fonction rechercherElement doit dans le pire des cas parcourir toute la liste pour retourner le dernier element. On est donc en O(n).

\medskip

La fonction supprimerElement, dans le meilleur des cas doit supprimer l'unique élement de la liste, cette suppression se fait en O(1). Sinon la fonction fait appel à la fonction rechercherElement, qui est en O(n). Le reste ne sont que des affectation. La fonction est donc en O(n) pour une liste de n éléments.

\medskip

La fonction supprimerListe, doit si la liste n'est pas nulle, parcourir les n élements de la liste et les supprimer. La fonction est donc en O(n). On a aussi prit en compte le cas particulier où la liste est vide, on est alors en 0(1) dans ce cas.   

\medskip

La fonction fusionnerListes doit, si une des 2 listes est vide directement retourner l'autre liste, ces opérations sont uniquement des opérations d'assignantion, elles se font en O(1). Sinon la fonction doit parcourir chaque liste et la réécrire dans la liste résultat. Il faut préciser que la fonction fera appel aux fonctions supprimerElement, supprimerListe & insererElement qui sont en O(n) dans le pire des cas, mais pas ici. En effet, on inserera toujours les valeurs en fin (queue) de liste pour la liste résultat, et on supprimera toujours les valeurs en tête de liste pour les deux listes à fusionner. De plus les listes supprimées étaient alors vides (donc suppression en O(1)). Ainsi, la suppression d'éléments sera toujours en O(1) ici et l'insertion dans la liste résultat sera elle aussi en O(1) car nous avons prévu ce cas particulier . Si la liste 1 possède n éléments et la liste 2 m éléments, la fonction doit réaliser min(n,m) fois les comparaisons, puis ajouter le reste de la liste non vide au résultat, on est donc en O(max(n,m)) dans le cas ou les listes ont des elements en commun. Cependant, si tous les elements d'une liste sont inferieurs a tous ceux de l'autre, alors on sera (et c'est le pire cas ici) en O(n+m) car il faudra gerer les deux listes a la suite.

\medskip

La fonction afficher doit parcourir toute la liste pour l'afficher, la fonction réalise donc n fois la boucle. La fonction est en O(n)

\medskip

La fonction menu permet d'afficher le menu. Cette fonction ne comporte aucune boucle ou appel récursif, elle est donc en 0(1).

\medskip

La fonction tableau\_vide permet de parcourir le tableau de 20 cases on est donc au maximum en O(20) donc la fonction est en O(1).

\medskip

Nous avons essayé de réduire au minimum la complexité de nos fonctions durant ce TP, et sommes parvenu à une complexité maximale de O(n+m) pour notre pire fonction (fusionnerListes). Nous avons en effet volontairement modifié nos fonctions de sorte à prendre en compte des cas particuliers et qui apparaissaient dans d'autres fonctions. Dans la fonction fusionnerListes, on fait par exemple appel aux fonctions d'insertion d'élement et de suppression d'élement, qui sont normalement en O(n) chacune. Voyant que nos insertions se faisaient toujours en queue de liste dans le cadre de fusionnerListes, nous avons rajouté ce cas particulier dans notre fonction d'insertion afin d'éviter de parcourir toute la liste inutilement. Nous n'avons pas eu besoin de modifier la fonction de recherche et de suppression d'élement car celles-ci étaient déjà en O(1) dans le cadre de la fonction fusionnerListes (car les suppressions sont toutes en tête de listes). Nous avons vérifié toutes nos fonctions afin d'optimiser autant que nous le pouvions leurs complexités, cela se traduit par une complexité plus faible mais un code plus lourd, car nécéssitant la prise en compte de cas particuliers. Cependant nous avons jugé que le code n'était pas trop alourdi et permettait un réel gain de complexité dans le cadre du TP.

\medskip

Concernant l'organisation du travail, nous avons réussi à bien nous répartir les tâches. Les fonctions creerElement et creerListe on été faites en TP, de même que le début de insererElement. Océane s'est chargée de débugger cette dernière et de la compléter, ainsi que de faire les fonctions rechercherElement & tableau\_vide. Louis s'est chargé de faire les fonctions supprimerElement, supprimerListe, fusionnerListes, afficher & menu. Ces fonctions ont été débuggées par Océane et le main.c fait par nous deux. La partie de test a aussi été faite par nous deux, et il en va de même pour le rapport.
Pour aider à l'organisation du travail, nous avions décidé de faire le TP sur Github afin de pouvoir y accéder facilement durant les vacances.

\end{document}




























